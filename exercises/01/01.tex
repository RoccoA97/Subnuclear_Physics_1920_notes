\providecommand{\main}{../../main}
\providecommand{\figpath}[1]{\main/../lessons/#1}
\documentclass[../../main/main.tex]{subfiles}



% Done in L10
\begin{document}

\chapter{Exercises}

\begin{exercise}{}{}
	\textbf{Calculate the average number of particles in an electromagnetic shower initiated by a 50 GeV photon, after 10, 13 and 20 cm of crossed iron.\\Hint: search for the radiation length of the iron on the PDG.}

	\medskip
	Searching on Particle Data Group\footnote{\url{http://pdg.lbl.gov/2010/AtomicNuclearProperties/HTML_PAGES/026.html}}, we find for \( e^- \):
	\begin{align}
		X_0^{\text{Fe}}	&=	1.757 \ \si{cm}	\\
		E_C^{\text{Fe}}	&=	21.68 \ \si{MeV}
		\label{eq:E01_1_1}
	\end{align}

	So, until the energy of the product particles is lower than the critical energy \( E_C \), we have:
	\begin{equation}
		N(x) = 2^{\frac{x}{X_0^{\text{Fe}}}}
		\label{eq:E01_1_2}
	\end{equation}
	We find these values:
	\begin{align}
		N(x = 10 \ \si{cm}) &\approx 52		\\
		N(x = 13 \ \si{cm}) &\approx 169	\\
		N(x = 20 \ \si{cm}) &\approx N(x \approx 19.6 \ \si{cm}) \approx 2306
	\end{align}
	We note that for \( x = 20 \ \si{cm} \), the critical energy has already been reached at \( x \approx 19.6 \ \si{cm} \), so the number of product particles has to be computed at this distance with Eq. \ref{eq:E01_1_2}.
\end{exercise}





\bigskip
\begin{exercise}{}{}
	\textbf{A muon of 100 GeV energy crosses without being absorbed a detector whose mass is mainly due to the hadronic calorimeter and to the muon detector. The thickness of the crossed material can be considered as a layer of 3 m of iron. Determine:}
	\begin{itemize}
		\item \textbf{What is the dominant energy loss process.}
		\item \textbf{The average loss of the muon inside the detector.}
	\end{itemize}
	\textbf{Hint: look at the energy loss picture of muons.}
\end{exercise}

\end{document}
