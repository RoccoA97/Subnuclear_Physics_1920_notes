\providecommand{\main}{../../main}
\providecommand{\figpath}[1]{\main/../lessons/#1}
\documentclass[../../main/main.tex]{subfiles}



% Done in L10
\begin{document}

\bigskip
\begin{exercise}{Electron inelastic scattering}{}
	\textbf{An electron with a \( E=20 \ \si{GeV} \) kinetic energy collides inelastically on a proton at rest. The electron is scattered at an angle \( \theta = 5 \ \si{°} \) with respect to its original direction and with an energy \( E'=12 \ \si{GeV} \). Calculate the effective mass of the final hadronic system.}

	\medskip
\end{exercise}





\bigskip
\begin{exercise}{Structure function}{}
	\textbf{The momentum distribution of the \( u \)-type quark in the proton can be parametrized by the formula:}\
	\begin{equation}
		F_u(x)
		\approx
		xu(x)
		=
		a(1-x)^2
		\label{eq:}
	\end{equation}
	\textbf{Determine the constant \( a \) with the assumption that the \( u \) quarks carry 33\% of the proton momentum.}

	\medskip
\end{exercise}





\bigskip
\begin{exercise}{Gluon structure function}{}
	\textbf{It is believed that the structure function describing the distribution of the gluon momentum inside the nucleons, \( g(x) \), strongly increases with decreasing \( x \). Estimate the number of gluons that would be possible to resolve with deep inelastic}
	\begin{equation}
		e + p \longrightarrow e + X
		\label{eq:}
	\end{equation}
	\textbf{collisions at \( Q^2=104 \ \si{GeV^2} \) at low \( x \) values (in the intervals \( 0.0001 \divisionsymbol 0.001 \), \( 0.001 \divisionsymbol 0.01 \), \( 0.01 \divisionsymbol 0.1 \)). Assume that at these \( Q^2 \) values the distribution function of the gluons is:}
	\begin{equation}
		xg(x)
		=
		0.36x^{-0.5}
		\label{eq:}
	\end{equation}

	\medskip
\end{exercise}

\end{document}
