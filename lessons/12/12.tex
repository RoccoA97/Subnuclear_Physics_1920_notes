\providecommand{\main}{../../main}
\providecommand{\figpath}[1]{\main/../lessons/#1}
\documentclass[../../main/main.tex]{subfiles}

\newdate{date}{21}{04}{2020}


\begin{document}

\chapter{Weak interactions}

\marginpar{ \textbf{Lecture 12.} \\  \displaydate{date}. \\ Compiled:  \today.}

Now we turn to the other subnuclear interaction, the weak interaction. The discussion of this topic starts from the observation of the medium lifetime of some unstable particles. QCD leads to a large spectrum of mesons and baryons, most of which are unstable with decay rates of the order of \( 100 \ \si{MeV} \), corresponding to lifetime of the order of \( 10^{-23} \ \si{s} \). However, the lightest particles of each type are more stable. MOst familiarly, the neutron is unstable by \( \beta \) decay:
\begin{equation}
	n
	\longrightarrow
	pe^-\bar{\nu}_e
	\label{eq:L12_BD}
\end{equation}
though it is very long-lived:
\begin{equation}
	\tau(n)
	=
	880 \ \si{s}
	\label{eq:L12_BDL}
\end{equation}
The great difference with the typical hadronic lifetimes suggest that those particles, like neutron, decays are due to another subnuclear interaction, different from the strong one.





\section{V-A Weak Theory}
Historically, it all started with the study of \( \beta \) decay. Before the discovery of neutrinos, it was observed that the spectrum of \( e^- \) was continuous, but this was not possible for a two-particle product decay, i.e. \( n \longrightarrow pe^- \). Pauli postulated the existence of a another invisible particle that enters in the products of \( \beta \) decay. Strange particles added other elements to the discussion on weak interaction. In fact, \( S \) (strangeness) is conserved in strong production of strange particles, but \( S \) must be violated in their decay. It was found:
\begin{align}
	K^0 &\xrightarrow{P=+1} \pi^+\pi^-	\\
	K^0 &\xrightarrow{P=-1} \pi^+\pi^-\pi^0
\end{align}
Parity violation was already confirmed by the experiment of Madame Wu with \( \text{Co}^{60} \).

Studying these decays, we observe that parity is maximally violated and this fact leads to the \textbf{V-A theory}.

Fermi proposed something like the diagram in Figure \ref{fig:L12_VAWTFP}.

%TODO p.232 (15.4)

This model proposed that all weak interaction matrix elements could be derived from a current-current interaction of the form:
\begin{equation}
	\mathcal{M}
	=
	\left\langle
	\frac{4G_F}{\sqrt{2}} j^{\mu+}_L j_{\mu L}^{-}
	\right\rangle
	\label{eq:}
\end{equation}
where:
\begin{align}
	j^{\mu+}_L	&= \mu^{\dag}_L \bar{\sigma}^{\mu} e_L + u^{\dag}_L \bar{\sigma}^{\mu} d_L + \dots \label{eq:L12_JML+} \\
	j^{\mu-}_L	&= e^{\dag}_L \bar{\sigma}^{\mu} \nu_L + d^{\dag}_L \bar{\sigma}^{\mu} u_L + \dots \label{eq:L12_JML-}
\end{align}
with \( e, \mu, u, d \) representing the lepton and quark fields. What is important to observe is that only the left-handed components of the Dirac field appear in Eqs. \ref{eq:L12_JML+} and \ref{eq:L12_JML-}

The name ``V minus A'' of the theory comes from rewriting:
\begin{equation}
	u^{\dag}_L \bar{\sigma}^{\mu} D_L
	=
	\bar{u} \gamma^{\mu} \qty(\frac{1 - \gamma^5}{2}) d
	=
	\frac{1}{2} \qty[\bar{u} \gamma^{\mu} d - \bar{u} \gamma^{\mu} \gamma^5 d]
	\label{eq:}
\end{equation}
which is a difference of the vector and the axial vector currents. Concerning the parameter \( G_F \), it is called \textbf{Fermi constant} and it has the dimensions of \( \si{GeV^{-2}} \):
\begin{equation}
	G_F
	=
	1.166 \cdot 10^{-5} \ \si{GeV^{-2}}
	\label{eq:}
\end{equation}



\subsection{Experimental tests of V-A theory}
Although V-A theory is quite simple, it makes a number of detailed and rather unexpected predictions for weak interaction processes that are confirmed by experiment.

\subsubsection*{Muon Decay}
The decay that we are considering now is:
\begin{equation}
	\mu^-
	\longrightarrow
	e^- + \bar{\nu}_e + \nu_{\mu}
	\label{eq:}
\end{equation}
The Feynman diagram of the process is in Figure \ref{fig:L12_MDFD}.

%TODO p.236 (15.19)

Using the various fermion fields to destroy and create initial and final particles, the matrix element is:
\begin{equation}
	\mathcal{M}
	\frac{4G_F}{\sqrt{2}}
	u^{\dag}_L(p_{\nu}) \bar{\sigma}^{\mu} u_L(p_{\mu})
	u^{\dag}_L(p_{e}) \bar{\sigma}_{\mu} v_L(p_{bar{\nu}})
	\label{eq:}
\end{equation}
From this result we can calculate teh energy spectrum:
\begin{equation}
	\dv{\Gamma}{E'}
	=
	\frac{G_F^2}{12\pi^3} m^2_{\mu} E'^2 \qty(8 - \frac{4E'}{m_{\mu}})
	\label{eq:}
\end{equation}
\begin{equation}
	\Gamma
	=
	\frac{1}{\tau}
	=
	\int \d{\Gamma}
	=
	\int \d{E} \ \dv{\Gamma}{E}
	=
	\frac{G^2_F m^5_{\mu}}{192\pi^3}
	\label{eq:}
\end{equation}
where \( E' \) is the energy of \( e^- \). A plot of the results is in Figure \ref{fig:L12_MDES}.

\begin{figure}[!h]
	\centering
	\includegraphics[width=0.75\textwidth]{\figpath{12}/12_images/MDES.png}
	\caption{\label{fig:L12_MDES} Energy spectrum of positrons emitted in muon decay \( \mu^+ \longrightarrow e^+ + \bar{\nu}_{\mu} + \nu_{e} \), and comparison to the V−A prediction.}
\end{figure}

The comparison of the total rate formula with the measured value of the muon lifetime gives a very accurate value of \( G_F \):
\begin{equation}
	G_F
	=
	(1.1663787 \pm 0.0000006) \cdot 10^{-5} \ \si{GeV^{-2}}
	\label{eq:}
\end{equation}

There is one more interesting aspect of the prediction for muon decay. At the endpoint \( x_e = 1 \), the configuration of the electron and the neutrinos is:

%TODO p.240 (15.43)

The \( \nu_{\mu} \) must be left-handed, the \( \bar{\nu}_e \) must be right-handed, and the electron must be left-handed. So the angular momenta of the neutrinos cancel and the total angular momentum in the final state is carried by the electron spin. This implies that the electron must be emitted in a direction opposite to the spin of the muon. In particular, the predicted distribution for electrons at the endpoint is:
\begin{equation}
	\dv{\Gamma}{\cos \theta}
	\sim
	\qty(1 - \cos \theta)
	\label{eq:}
\end{equation}
with a maximum when the electron is moving opposite to the muon spin and a zero when the electron is parallel to the muon spin. This prediction was checked explicitly in an experiment at the TRIUMF laboratory in Vancouver, Canada, in which \( \mu^+ \)s from pion decay were stopped in an absorber and then allowed to decay. Muons from pion decay are perfectly polarized, a magnetic field was used to precess the spins of the stopped muons, and the decay electrons were counted as a function of time. The signal was seen to oscillate as the muons precess, as we can see from the data in Figure \ref{fig:L12_PDTMOS}.

\begin{figure}[!h]
	\centering
	\includegraphics[width=0.5\textwidth]{\figpath{12}/12_images/PDTMOS.png}
	\caption{\label{fig:L12_PDTMOS} Signal rates as a function of time, as the muon spin is precessed in a magnetic field, in the TRIUMF measurement of the correlation of the positron direction with the muon spin.}
\end{figure}



\subsubsection*{Pion decay}
The processes we are studying now is:
\begin{align}
	\pi^- &\longrightarrow \mu^-\bar{\nu}_{\mu}	\\
	\pi^- &\longrightarrow e^-\bar{\nu}_{e}
\end{align}
If we look at the branching ratio:
\begin{equation}
	\frac{\Gamma(\pi^- &\longrightarrow e^-\bar{\nu}_{e})}{\Gamma(\pi^- &\longrightarrow \mu^-\bar{\nu}_{\mu})}
	=
	1.23 \cdot 10^{-4}
	\label{eq:}
\end{equation}
%TODO ho scritto molte cazzate
Why is this happening? The reason hides in the characteristics of the decay. The pion has spin 0, so in the decay we have in one side the electron, in the other side the antineutrino, which is right-handed. So the electron should be right-handed in order to have the momentum conserved. So the electron has the wrong helicity and the same is true also for the muon. What is making the difference? The difference comes from the mass. It's more difficult for a massive particle to be in a wrong state. As the energy of the particle increases, the particles become more left-handed or right-handed. So for the massive lepton we have just one possibility, for the muon it is easier to be in the wrong helicity state


\end{document}
