\providecommand{\main}{../../main}
\providecommand{\figpath}[1]{\main/../lessons/#1}
\documentclass[../../main/main.tex]{subfiles}

\newdate{date}{12}{05}{2020}


\begin{document}

\marginpar{ \textbf{Lecture 18.} \\  \displaydate{date}. \\ Compiled:  \today.}

\section{Cabibbo Theory and CKM matrix}
% p.284
Particle masses and quark mixing have the same origin in the Standard Model. In the Electroweak theory, particles are classified in left-handed doublets. The Higgs field \( \varphi \) was introduced to explain why fermions and bosons have masses different from zero.
\begin{equation}
	\Delta \mathcal{L}
	=
	- y_e^{ij} L_a^{+i} \varphi_Q e^{J}_r - y_{\alpha}^{ij} Q^{+i}_Q \varphi_a d^{J}_R - y_{u}^{ij} Q^{+}_{Q} \varepsilon \varphi^*_b u{J}_R + \text{h.c.}
	\label{eq:}
\end{equation}
where \( y_f \) are the Yukawa couplings and:
\begin{equation}
	L =
	\begin{pmatrix}
		\nu_L \\
		e_L
	\end{pmatrix}
	\qquad
	Q =
	\begin{pmatrix}
		u_L \\
		d_L
	\end{pmatrix}
	\label{eq:}
\end{equation}
If we replace the Higgs field by its vacuum expectation value
\begin{equation}
	\varphi \longrightarrow
	\begin{pmatrix}
		Q \\
		\frac{v}{\sqrt{2}}
	\end{pmatrix}
	\label{eq:}
\end{equation}
the Lagrangian gives:
\begin{equation}
	\Delta \mathcal{L}
	=
	- \frac{y_e v}{\sqrt{2}} e^\dag_L e_R
	- \frac{y_d v}{\sqrt{2}} d^\dag_L d_R
	- \frac{y_u v}{\sqrt{2}} u^\dag_L u_R
	+ \text{h.c.}
	\label{eq:}
\end{equation}
Comparing this equation, we see that it has just the structure of mass terms for the \( e \), \( d \), and \( u \). Then:
\begin{equation}
	m_f
	=
	y_f \frac{v}{\sqrt{2}}
	\label{eq:}
\end{equation}
for all the three species. This is what we want to measure.

The fermion status couples also to the mediators, \( \gamma \), \( Z^0 \), \( W^{\pm} \). In order to describe fermions-mediators interaction, we need to diagonalize the couplings for the fermions, \( y^{ij}_f \). In the case of leptons, the change in the Lagrangian is something like:
\begin{equation}
	\Delta \mathcal{L}
	\sim
	e^\dag_R (i\sigma D) e_R
	\label{eq:}
\end{equation}
Leptons interact only with the same generation. This has as outcome the lepton number conservation.

For quarks, for \( Z^0 \), \( \gamma \): the \( \Delta \mathcal{L} \) ss similar of the lepton one.
For \( W^{\pm} \) %TODO

Interactions with \( W^{\pm} \) can happen with different generations.

\( V_\mathrm{CKM} \) elements have to be evaluated.

\end{document}
