% Energy level diagrams - illustrating Hund's rule
% Author: Henri Menke
\documentclass[tikz, border=10pt]{standalone}
\usepackage{siunitx}
\usetikzlibrary{shapes.callouts}
\tikzset{
  level/.style   = { ultra thick },
  connect/.style = { dashed, red },
  notice/.style  = { draw, rectangle callout, callout relative pointer={#1} },
  label/.style   = { text width=2cm },
  trans/.style   = { dashed, blue }
}
\begin{document}
\begin{tikzpicture}
	% PSEUDOSCALAR MESONS
	\draw[level] (1,0) -- node[above] {\small\( \pi^{-} \)} (2,0);
	\draw[level] (3,0) -- node[above] {\small\( \pi^{0} \)} (4,0);
	\draw[level] (5,0) -- node[above] {\small\( \pi^{+} \)} (6,0);

	\draw[level] (0,2) -- node[above] {\small\( K^{-} \)}		(1,2);
	\draw[level] (2,2) -- node[above] {\small\( \bar{K}^{-} \)}	(3,2);
	\draw[level] (4,2) -- node[above] {\small\( K^{0} \)}		(5,2);
	\draw[level] (6,2) -- node[above] {\small\( K^{+} \)}		(7,2);

	\draw[level] (3,4) -- node[above] {\small\( \eta \)}		(4,4);

	\draw[level] (3,6) -- node[above] {\small\( \eta' \)}		(4,6);

	\draw node at (8,0) {\small140};
	\draw node at (8,2) {\small498};
	\draw node at (8,4) {\small548};
	\draw node at (8,6) {\small958};



	% VECTOR MESONS
	\draw[level] (11,0) -- node[above] {\small\( \rho^{-} \)} (12,0);
	\draw[level] (13,0) -- node[above] {\small\( \rho^{0} \)} (14,0);
	\draw[level] (15,0) -- node[above] {\small\( \rho^{+} \)} (16,0);

	\draw[level] (13,0.8) -- node[above] {\small\( \omega^{0} \)} (14,0.8);

	\draw[level] (10,2) -- node[above] {\small\( K^{+-} \)}			(11,2);
	\draw[level] (12,2) -- node[above] {\small\( \bar{K}^{+0} \)}	(13,2);
	\draw[level] (14,2) -- node[above] {\small\( K^{+0} \)}			(15,2);
	\draw[level] (16,2) -- node[above] {\small\( K^{*+} \)}			(17,2);

	\draw[level] (13,4) -- node[above] {\small\( \phi^{0} \)} (14,4);

	\draw node at (18,0) 	{\small770};
	\draw node at (18,0.8) 	{\small781};
	\draw node at (18,2) 	{\small892};
	\draw node at (18,4) 	{\small1020};


\end{tikzpicture}
\end{document}
