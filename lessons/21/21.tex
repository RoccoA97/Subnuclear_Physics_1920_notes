\providecommand{\main}{../../main}
\providecommand{\figpath}[1]{\main/../lessons/#1}
\documentclass[../../main/main.tex]{subfiles}

\newdate{date}{20}{05}{2020}


\begin{document}

\marginpar{ \textbf{Lecture 21.} \\  \displaydate{date}. \\ Compiled:  \today.}

%TODO manca la parte precedente

The decay is shifted forward in time for an \( \bar{B}^0 \) and backward in time for an initial \( B^0 \). The asymmetry is predicted to have a time-dependence governed by \( \delta m \) with amplitude \( \sin 2\beta \). For the process \( B^0/\bar{B}^0 \rightarrow J/\psi K^0_L \), the relative minus sign in the decay amplitudes from \( B^0 \) and \( \bar{B}^0 \) becomes a plus sign and so the asymmetry takes the minus sign. The angle \( \beta \) is the phase angle taken directly from CKM matrix, without corrections due to strong interaction.

The first thing to do in order to understand the time-dependent asymmetry is to find a way to produce a sufficient quantity of \( B^0 \) and \( \bar{B}^0 \) mesons. This can be done \( e^+e^- \) annihilation, which leads to a state with \( J=1 \). This means that for the production of spin 0 mesons, the two mesons are in an \( L=1 \) wavefunction, antisymmetric in the other meson quantum numbers. In particular, the \( B \) mesons go outward from the production point and, after some time, one of the mesons decays. If it decays to an \( e^+ \) or a \( \mu^+ \), this event tags this meson (at this time) as a \( B^0 \). The other meson must then be a \( \bar{B}^0 \). This state propagates for an additional time \( \Delta t \), possibly mixing to \( B^0 \) during that time, and then decays to the observed final state. Note that the relative time \( \Delta t \) might be negative if the leptonic decay takes place after the selected exclusive decay.

%TODO (19.61) p.304
%TODO Fig. 19.4

The lifetime of the \( B \) meson is about \( 1.5 \ \si{ps} \), so it is difficult to measure the decay time directly. A possibility is to construct an asymmetric colliding beam accelerator, in which the \( e^+e^- \) center of mass frame is moving with respect to the lab. The boost of the center of mass is approximately \( v/c \sim 0.5 \). Therefore, two \( B \) decays would be separeted by about \( 200 \ \si{\mu m} \), which is a resolvable distance for a silicon tracking detector which pinpoints the decay vertices.

In the late 1990's, two asymmetric \( e^+e^- \) colliders were constructed, one at SLAC (\( 9.0 \ \si{GeV} \ e^- \ \times 3.1 \ \si{GeV} \ e^+ \)), for the BaBar experiment, and one at KEK in Tsukuba, Japan (\( 8.0 \ \si{GeV} \ e^- \ \times 3.5 \ \si{GeV} \ e^+ \)), for BELLE experiment. In 2001, both experiments observed the \( CP \)-violating asymmetry in \( B^0 \rightarrow J/\psi K^0_S \).

% TODO BELLE2

In Figure \ref{fig:L20_BABAR} it is presented the displacements of the decay distributions for \( B^0 \rightarrow J/\psi K^0 \) and \( \bar{B}^0 \rightarrow J/\psi K^0 \) measured by the BaBar experiment. The distributions are labelled by the tagging \( B \) meson, so the points labeled ``\( B^0 \) tags'' indicate \( \bar{B}^0(\tau) \) decays, and vice versa. The distributions for \( B^0 \) and \( \bar{B}^0 \) are shifted substantially with respect to one another, in just the directions predicted below. The shifts are in the opposite directions for \( K^0_L \) instead of \( K^0_S \) in the final state. The current best value of \( \beta \) from this measurement is:
\begin{equation}
	\sin 2\beta
	=
	0.679 \pm 0.20
	\label{eq:}
\end{equation}
that is, \( \beta = 21\text{°} \). This is indeed a large \( CP \)-violating effect.

\end{document}
