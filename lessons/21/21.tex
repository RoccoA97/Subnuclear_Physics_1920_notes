\providecommand{\main}{../../main}
\providecommand{\figpath}[1]{\main/../lessons/#1}
\documentclass[../../main/main.tex]{subfiles}

\newdate{date}{20}{05}{2020}


\begin{document}

\marginpar{ \textbf{Lecture 21.} \\  \displaydate{date}. \\ Compiled:  \today.}

%TODO manca la parte precedente

The decay is shifted forward in time for an \( \bar{B}^0 \) and backward in time for an initial \( B^0 \). The asymmetry is predicted to have a time-dependence governed by \( \delta m \) with amplitude \( \sin 2\beta \). For the process \( B^0/\bar{B}^0 \rightarrow J/\psi K^0_L \), the relative minus sign in the decay amplitudes from \( B^0 \) and \( \bar{B}^0 \) becomes a plus sign and so the asymmetry takes the minus sign. The angle \( \beta \) is the phase angle taken directly from CKM matrix, without corrections due to strong interaction.

The first thing to do in order to understand the time-dependent asymmetry is to find a way to produce a sufficient quantity of \( B^0 \) and \( \bar{B}^0 \) mesons. This can be done \( e^+e^- \) annihilation, which leads to a state with \( J=1 \). This means that for the production of spin-0 mesons, the two mesons are in an \( L=1 \) wavefunction, antisymmetric in the other meson quantum numbers. In particular, the \( B \) mesons go outward from the production point and, after some time, one of the mesons decays. If it decays to an \( e^+ \) or a \( \mu^+ \), this event tags this meson (at this time) as a \( B^0 \). The other meson must then be a \( \bar{B}^0 \). This state propagates for an additional time \( \Delta t \), possibly mixing to \( B^0 \) during that time, and then decays to the observed final state. Note that the relative time \( \Delta t \) might be negative if the leptonic decay takes place after the selected exclusive decay. These processes are schematized in Figure \ref{fig:L21_EPJP}.

\begin{figure}[!h]
	\centering
	\includegraphics[width=0.5\textwidth]{\figpath{21}/21_images/EPJP.png}
	\caption{\label{fig:L21_EPJP} Production of \( B^0 \) and \( \bar{B}^0 \) mesons through \( e^+e^- \) annihilation and decay products.}
\end{figure}

The lifetime of the \( B \) meson is about \( 1.5 \ \si{ps} \), so it is difficult to measure the decay time directly. A possibility is to construct an asymmetric colliding beam accelerator, in which the \( e^+e^- \) center of mass frame is moving with respect to the lab. The boost of the center of mass is approximately \( v/c \sim 0.5 \). Therefore, two \( B \) decays would be separeted by about \( 200 \ \si{\mu m} \), which is a resolvable distance for a silicon tracking detector which pinpoints the decay vertices.

In the late 1990's, two asymmetric \( e^+e^- \) colliders were constructed, one at SLAC (\( 9.0 \ \si{GeV} \ e^- \ \times 3.1 \ \si{GeV} \ e^+ \)), for the BaBar experiment, and one at KEK in Tsukuba, Japan (\( 8.0 \ \si{GeV} \ e^- \ \times 3.5 \ \si{GeV} \ e^+ \)), for BELLE experiment. In 2001, both experiments observed the \( CP \)-violating asymmetry in \( B^0 \rightarrow J/\psi K^0_S \).

In Figure \ref{fig:L21_BABAR} it is presented the displacements of the decay distributions for \( B^0 \rightarrow J/\psi K^0 \) and \( \bar{B}^0 \rightarrow J/\psi K^0 \) measured by the BaBar experiment. The distributions are labelled by the tagging \( B \) meson, so the points labeled ``\( B^0 \) tags'' indicate \( \bar{B}^0(\tau) \) decays, and vice versa. The distributions for \( B^0 \) and \( \bar{B}^0 \) are shifted substantially with respect to one another, in just the directions predicted below. The shifts are in the opposite directions for \( K^0_L \) instead of \( K^0_S \) in the final state.

\begin{figure}[!h]
	\centering
	\includegraphics[width=0.75\textwidth]{\figpath{21}/21_images/BABAR.png}
	\caption{\label{fig:L21_BABAR} Proper time distribution of \( B^0\bar{B}^0 \rightarrow J/\psi K^0 \) decays at the \( \Upsilon (4\text{S}) \), measured by the BaBar experiment at the PEP-II collider at SLAC. Panel (a) shows the decay distributions for \( B^0\bar{B}^0 \rightarrow J/\psi K^0_S \). Panel (b) shows the rate asymmetry. Panel (c) shows the decay distributions for \( B^0\bar{B}^0 \rightarrow J/\psi K^0_L \). Panel (d) shows the corresponding rate asymmetry.}
\end{figure}

The current best value of \( \beta \) from this measurement is:
\begin{equation}
	\sin 2\beta
	=
	0.679 \pm 0.20
	\label{eq:}
\end{equation}
that is, \( \beta = 21\text{°} \). This is indeed a large \( CP \)-violating effect.

Concerning the angles \( \alpha \) and \( \gamma \), they can also be measured by observable parameters of \( B \) decays. The angle \( \alpha \) is given by time-dependent asymmetries in \( B \) decay to light quarks:
\begin{align}
	B^0 &\longrightarrow \pi^+\pi^-	\\
	B^0 &\longrightarrow \pi^{\pm}\rho^{\mp}	\\
	B^0 &\longrightarrow \rho^+\rho^-
\end{align}
The angle \( \gamma \) can be extracted from asymmetries in \( B \) decays to \( DK \). These constraints are shown in Figure \ref{fig:L21_CKMC},vtogether with constraints from the value of \( \abs{V_{ub}} \), the values of the \( B^0 \)-\( \bar{B}^0 \) mixing amplitude, the value of \( B^0_s \)-\( \bar{B}^0_s \) mixing amplitude, and the value of \( \varepsilon \) from the neutral \( K \) system. In the Standard Model, all of these parameters must be consistent with a common value of \( (\rho + i\eta) \).

\begin{figure}[!h]
	\centering
	\includegraphics[width=0.75\textwidth]{\figpath{21}/21_images/CKMC.png}
	\caption{\label{fig:} Constraints on the CKM parameters \( (\rho,\eta) \) from measurements of \( CP \) violation, showing the fit by the CKMFitter collaboration.}
\end{figure}

So, any quantum field theory is invariant under \( CPT \), so \( CP \) violation implies \( T \) violation. However, it is interesting to ask whether one can directly see \( T \) violation in heavy quark decays.
The BaBar experiment demonstrated this in the following way: We have seen
that, in \( e^+e^- \) annihilation, \( B \) mesons are produced as pairs in a quantum coherent wavefunction. The decay of one meson breaks the coherence, identifying one meson of the pair as a \( B^0 \) or a \( \bar{B}^0 \), for a leptonic decay, or as a \( CP=+ \) or \( CP=- \) state (\( B_+ \) or \( B_- \)), for a decay to a \( CP \) eigenstate. We can then pick out events in which the leptonic decay happens first, followed by time evolution to a \( CP \) eigenstate, and also events in which the \( CP \) decay happens first, followed by time evolution to a state with a definite leptonic decay.
If the equations of motion of nature were \( T \) symmetric, the rates for time evolution in the two directions would be equal. They are not. The asymmetries between the rates for pairs of time-reversed processes (e.g., \( B^0 \rightarrow B_- \) vs. \( B_- \rightarrow B^0 \)) are shown in Figure \ref{fig:L21_TRVA}. Note that the asymmetries reverse when one changes from \( B^0 \) to \( \bar{B}^0 \) and from even to odd \( CP \), consistent with the physics described above. This is the most direct
evidence that the equations of nature violate time reversal invariance.

\begin{figure}[!h]
	\centering
	\includegraphics[width=0.75\textwidth]{\figpath{21}/21_images/TRVA.png}
	\caption{\label{fig:L21_TRVA} Time reversal violating asymmetries measured as a function of proper time by the BaBar experiment at the PEP-II collider. The four panels refer to the transitions: \( \bar{B}^0 \rightarrow B_- \) (a), \( B_+ \rightarrow B^0 \) (b), \( \bar{B}^0 \rightarrow B_+ \) (c), \( B_- \rightarrow B^0 \) (d).}
\end{figure}

\end{document}
