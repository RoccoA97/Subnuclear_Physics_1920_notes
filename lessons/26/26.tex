\providecommand{\main}{../../main}
\providecommand{\figpath}[1]{\main/../lessons/#1}
\documentclass[../../main/main.tex]{subfiles}

\newdate{date}{09}{06}{2020}


\begin{document}

\marginpar{ \textbf{Lecture 26.} \\  \displaydate{date}. \\ Compiled:  \today.}

\section{Future of Subnuclear Physics}

\subsection{Open questions}

\subsubsection{How do the pieces of the Standard Model fit together?}
The Standard Model Lagrangian contains 3 different gauge symmetry groups and 15 different fermion representations. In a final theory, shouldn’t there just be one gauge symmetry and one type of matter?

The idea that there is a single fundamental gauge group, and that this symmetry group is spontaneously broken at short distances to \( SU(3) \times SU(2) \times U(1) \), is called grand unification. The grand unified theory is a Yang-Mills theory with a single coupling constant. This seems to contradict our knowledge that the \( SU(3) \), \( SU(2) \) and \( U(1) \) coupling constants are very different. It is not hard to imagine that the three Standard Model couplings meet at some small distance scale where the spontaneous breaking takes place. Expressed as an energy scale, the location of the symmetry breaking turns out to be close to \( 10^{16} \ \si{GeV} \).


\subsubsection{Why do the quark and lepton masses vary over such a large range?}
One of the most striking features of the Standard Model is that it accomodates a top quark of mass about \( 170 \ \si{GeV} \) and an \( u \) quark whose mass is \( 10^5 \) times smaller. In each case, the mass of the quark is given by the size of the corresponding Higgs boson Yukawa coupling. These coupling constants are inputs to the Standard Model.

Actually, the Standard Model does not even give insight into what might seem to be an easier question: Is the top quark a ``heavy'' quark, while the other quarks and leptons have more ordinary values, or does the top quark have a ``normal'' value for its mass, while the masses of all other fermions are for some reason suppressed? Since the mass of the top quark is of the same order of magnitude as the masses of the \( W \), \( Z \), and Higgs bosons, it is tempting to say that the top quark mass is of the expected magnitude for fermion masses.


\subsubsection{Why is the weak interaction gauge symmetry \( SU(2) \times U(1) \) spontaneously broken?}
Many theories have been put forward to provide an underlying explanation for the shape of the potential energy of the Higgs field and its preference for an asymmetric vacuum state.

If one dismisses the Standard Model Higgs potential as too simplistic and looks for a physics-based explanation for the symmetry-breaking of \( SU(2) \times U(1) \), it is unavoidable that there exist new fundamental interactions still to be discovered at higher energies or shorter distances.


\subsubsection{Why is the universe not uniform, but, rather, full of structure?}
We have strong evidence from astro-physics that the universe originated in a state of very high temperature, the ````Big Bang''. At this temperature, all of the particles of the Standard Model would have been created by thermal pair production and would have been present in large numbers. As the universe cooled, particles and antiparticles would have annihilated, leaving us with the universe with empty space and clumps of stable matter that we see today.

In principle, we should be able to understand the composition of the universe and the growth of cosmic structure such as galaxies and galaxy clusters by taking the high-temperature state as the initial condition and evolving from that point using the equations of the Standard Model. To grow these structures, we need, first, seeds given by small density inhomogeneities, and, second, a mechanism for these inhomogeneities to grow as the universe evolves. The growth of structure can be accomplished by gravity, with small excesses of matter attracting additional matter and growing into large density excesses. But how did the small excess arise in the first place?

In 1981, Alan Guth proposed an explanation of these features from a model called the inflationary universe. In this picture, the universe began its evolution containing a scalar field with a very large positive value of its potential energy. The coupling of this scalar field to gravity leads to an exponential expansion of every small patch of the universe. In the inflationary model, a patch of a few \( \si{cm} \) in size expands to the size of the current universe. Inflation is terminated by the transition of the scalar field to its ground state with much lower potential energy. The conversion of the original potential energy into heat provides the thermal energy of the Big Bang. So, already, in order to create the correct initial conditions for the universe, we need to postulate at least one additional scalar field that is not contained in the Standard Model.


\subsubsection{Why does the universe contain more matter than antimatter?}
In principle, the universe could have begun with an initial small excess of matter over antimatter. When quarks and antiquarks annihilated as the temperature of the universe fell below \( 1 \ \si{GeV} \), the excess quarks would have been left over. An initial excess of only 1 part in \( 10^{10} \) is needed. However, if we accept the idea that the initial conditions of the universe came from a period of inflation, this explanation cannot be valid. The dramatic expansion required by inflation emptied the universe of particles and set the initial matter-antimatter asymmetry to zero. Then the needed asymmetry must have developed in the evolution of the universe after the Big Bang.

In principle, we can compute the evolution of the components of the universe from the equations of the Standard Model. In particular, in order to create a nonzero asymmetry between the numbers densities of matter and antimatter, these equations must be asymmetric between matter and antimatter, violating \( CP \) symmetry.

The Standard Model contains a \( CP \)-violating parameter, the CKM phase, and this does produce a matter-antimatter asymmetry in the early universe. However, it turns out that this asymmetry is too small by a factor of \( 10^8 \) to produce today's known matter density of the universe. The influence of the CKM phase in the early universe is proportional to the product of the light quark Yukawa couplings, and so is very small. Then, another source of \( CP \) violation is needed. One interesting suggestion is that the new source of \( CP \) violation is the Majorana mass for right-handed neutrinos. This neutrino mechanism for the production of a baryon asymmetry is called leptogenesis.


\subsubsection{What is the ``dark matter'' of the universe?}
The internal dynamics of galaxies and of clusters of galaxies require that these objects contain large amounts of invisible and weakly interacting matter that interacts gravitationally with the atoms. These observations are corroborated by observations of the cosmic microwave background. In addition, we now know that the growth of structure in the universe since the Big Bang would be too slow if the gravitational clumping of matter were driven by the gravity of atomic matter only. From all of these sources, we deduce that 85\% of the nonrelativistically moving matter in the universe is of this new type, called dark matter.

Dark matter must be made of particles, but those particles cannot be any of the particles in the Standard Model. For particle physicists, this is a supreme embarrassment. With all of our knowledge, we are ignorant of the origin of most of the matter in the universe.


\subsubsection{What is the ``dark energy'' of the universe?}
In 1998, another mysterious ingredient was added to this picture. From measurements of the red shifts of distant supernovae, two groups of observers demonstrated that the universe is in a phase of exponential expansion even today. This expansion would be accounted for by a small potential energy in each unit volume of empty space, due either to another new scalar field or to quantum effects from known and unknown elementary particles. This ingredient is called dark energy.


\subsubsection{Are there higher symmetries of nature that lead to new particles and interactions?}
Many of the questions that we have already considered, in particular, the missing explanations for \( SU(2) \times U(1) \) symmetry breaking, the generation of the observed matter-antimatter asymmetry, and the dark matter of the universe, call for new particles and interactions that must be added to the Standard Model. It is a very attractive idea that these new interactions might have a fundamental basis. Perhaps, by extending the space-time symmetries of the Standard Model, these new ingredients might naturally appear. There is in fact a unique extension of the group of space-time symmetries, the translations, rotations, and Lorentz boosts. This extended group adds supersymmetries, operations that change the total spin by \( \frac{1}{2} \) unit, transforming bosons into fermions and fermions into bosons.


\subsubsection{How does gravity fit together with the Standard Model?}
A truly final theory should incorporate all of the known forces of nature, including gravity. The description of gravity at the classical level is given by Einstein's theory of general relativity. This is a very well-tested theory. General relativity has a quantum version. In that theory, the gravitational force is mediated by a massless spin-2 particle, the graviton. This quantum theory of gravity has a formalism of Feynman diagrams, similar to those of the Standard Model, from which scattering amplitudes can be computed. From this point of view, the quantum theory of gravity can simply be appended as another component of the Standard Model, though its unification with the other forces is not explained.

There is, however, a more serious problem with quantum gravity. The quantum theory of gravity differs from the gauge theories of the Standard Model in having a dimensionful coupling constant, Newton's constant \( G_{\mathrm{N}} \). Expressed as an energy, Newton's constant gives a mass parameter, the Planck scale \( m_{\mathrm{Pl}} \), equal to \( 10^{19} \ \si{GeV} \). At energies below \( m_{\mathrm{Pl}} \), the Feynman diagram expansion makes sense, but at energies approaching \( m_{\mathrm{Pl}} \) the theory becomes strongly coupled and our methods of calculation fail catastrophically. Speaking roughly, Einstein's theory predicts that, at distances of \( 1 / m_{\mathrm{Pl}} \), spacetime itself becomes singular due to quantum fluctuations.


\subsubsection{Is space-time a fundamental concept that will survive in the final theory?}
The difficulties of formulating a quantum theory of gravity valid at all energies suggest the idea that continuous space-time itself is an approximate notion that will be replaced in a more fundamental theory. At currently explored energies, up to the energies probed by the LHC, Lorentz invariance, the continuity of space, and the locality of quantum field theory interactions are all extremely well tested. But it is a long way from \( \si{TeV} \) energies to the Planck scale. Many surprises and new concepts might make themselves apparent between here and there.



\subsection{Future experiments on neutrino}
Some gigantic detectors with liquid medium are under way:
\begin{itemize}
	\item liqquid scintillator: \textbf{Juno} and \textbf{SNO+} are in construction;
	\item water: \textbf{Hyper-Kamiokande} and \textbf{IceCube Gen 2};
	\item liquid argon: \textbf{Dune} is approved and partially funded.
\end{itemize}
Their structure is showed in Figures \ref{fig:L26_JUNO}, \ref{fig:L26_HK} and \ref{fig:L26_DUNE}.



\subsection{Future accelerators}
New accelerators projects have been proposed (showed in Figure \ref{fig:L26_FCCMC}):
\begin{itemize}
	\item \textbf{FCC-ee}: it proposes to study Higgs couplings with high precision but it can not measure Higgs potential. It would produce \( e^+e^- \) collisions at \( \sqrt{s} = 40 \divisionsymbol 400 \ \si{GeV} \).
	\item \textbf{FCC-hh}: it would produce \( pp \) collisions at \( \sqrt{s} = 100 \ \si{TeV} \). \( 100 \ \si{km} \) of tunnel is needed and it will be challenging to construct magnets. Moreover, parton PDFs at that energies can limit the precision measurements. However, it would allow to fully investigate the Higgs sector.
	\item \textbf{Muon collider at very high \( \sqrt{s} \)}: it is a portal to new physics study since at high energies it is a \( W^+W^- \) collider and Higgs coupling and potential can be measured. However, there are several challenges to face:
	\begin{itemize}
		\item[\( \triangleright \)] Muons are tertiary particles and to have enough we need to start with intense proton beams on target to produce pions that decay to muons.
		\item[\( \triangleright \)] Muons decay, acceleration and beam formation must be very fast, a \( 1.5 \ \si{TeV} \) muon travels \( 9300 \ \si{km} \) in a lifetime.
		\item[\( \triangleright \)] Muons are born with a large 6D phase space, new technology to ``cool'' them is needed.
	\end{itemize}
\end{itemize}



\begin{figure}[!h]
	\centering
	\includegraphics[width=0.8\textwidth]{\figpath{26}/26_images/JUNO.png}
	\caption{\label{fig:L26_JUNO} Juno experiment.}
\end{figure}

\begin{figure}[!h]
	\centering
	\includegraphics[width=0.9\textwidth]{\figpath{26}/26_images/HK.png}
	\caption{\label{fig:L26_HK} Hyper-Kamiokande experiment.}
\end{figure}

\begin{figure}[!h]
	\centering
	\includegraphics[width=0.9\textwidth]{\figpath{26}/26_images/DUNE.png}
	\caption{\label{fig:L26_DUNE} Dune experiment.}
\end{figure}

\begin{figure}[!h]
	\begin{minipage}[t]{0.47\linewidth}
		\centering
		\includegraphics[height=0.5\textwidth]{\figpath{26}/26_images/FCC.png}
	\end{minipage}
	\hspace{0.05\linewidth}
	\begin{minipage}[t]{0.47\linewidth}
		\centering
		\includegraphics[height=0.5\textwidth]{\figpath{26}/26_images/MC.png}
	\end{minipage}
	\caption{\label{fig:L26_FCCMC} Left: project of Future Circular Collider (FCC), in comparison with LHC. Right: project of muon collider at very high \( \sqrt{s} \).}
\end{figure}

\end{document}
