\providecommand{\main}{../../main}
\providecommand{\figpath}[1]{\main/../lessons/#1}
\documentclass[../../main/main.tex]{subfiles}

\newdate{date}{27}{05}{2020}


\begin{document}

\marginpar{ \textbf{Lecture 23.} \\  \displaydate{date}. \\ Compiled: \today.}

%TODO p.315

There is an oscillation between the flavor eigenstates with an oscillation length:
\begin{equation}
	L
	=
	4\pi \frac{E}{\Delta m^2}
	=
	(2.48 \ \si{m}) \frac{E \ (\si{MeV})}{\Delta m^2 \ \si{eV^2}}
	\label{eq:}
\end{equation}
The conclusion is quite surprising. We can detect the presence of small neutrino masses if the neutrinos also exhibit flavor mixing. Then the effect of the mass term is to generate a flavor oscillation as a function of the distance from the neutrino source. For \( \si{MeV} \) neutrinos with \( 10^{-2} \ \si{eV} \) masses or for \( \si{GeV} \) neutrinos with \( 10^{-1} \ \si{eV} \) masses, the length scale of the oscillation can be \( \si{km} \).

This is just the opposite of the way that we determine the masses and weak interaction flavor mixing among quarks. For quarks, we observe the particles as mass eigenstates, inside hadrons of definite mass. Decays through the weak interaction show that the mass eigenstates are linear combinations of weak interaction eigenstates. For neutrinos, the primary way that we observe the particles is through weak interaction decay. Then we characterize the neutrino eigenstates according to their weak interaction properties. It is the flavor mixing as the neutrinos travel that demonstrates that there is a mass eigenstate basis, with different masses for the three neutrinos, that is different from the flavor basis.





\section{Neutrino mixing evidence}
Now that we know how to look for neutrino mass, we can discuss the experimental evidence that the neutrino masses are indeed nonzero.

The first clear evidence for neutrino flavor mixing, and, thus, for neutrino mass, came in the study of the neutrinos produced in cosmic ray interactions in the atmosphere. These were observed in underground water Cherenkov detectors originally built to look for proton decay. It was observed that the flux of \( \nu_{e} \) from atmospheric interactions was close to the predictions, while the flux of \( \nu_{\mu} \) was too small by a factor of 2.

In 1998, the SuperKamiokande experiment, a very large water Cherenkov detector in the Kamioka mine in Japan, resolved this question by observing the directions of \( \nu_{\mu} \)'s from their conversion to muons in charge-changing interactions. The downward-going \( \nu_{\mu} \) were present with a flux that was essentially unsuppressed, while upward-going \( \nu_{\mu} \), created on the other side of the earth, were highly supressed. For \( \nu_{e} \), the ratio of the predicted to the observed flux was independent of direction.

%TODO data in fig. 20.2

This strongly indicated a flavor mixing \( \nu_{\mu} \leftrightarrow \nu_{\tau} \) on the scale of the Earth's diameter. The mixing angle was consistent with a maximal value:
\begin{equation}
	\sin^2(2\theta)
	=
	1
	\label{eq:}
\end{equation}
This flavor mixing has since been confirmed by accelerator experiments that create beams of \( \nu_{\mu} \) at \( \si{GeV} \) energies and detect the neutrinos over a long path length. The current best values of the oscillation parameters are:
\begin{align}
	\Delta m^2 &= (2.43 \pm 0.08) \cdot 10^{-3} \ \si{eV^2} = (5 \cdot 10^{-2} \ \si{eV})^2	\\
	\sin^2\theta &= 0.386 \pm 0.023
\end{align}

The mass of the \( \nu_{e} \) is related to another long-standing anomaly in neutrino physics. In the 1960's, John Bahcall suggested testing the mechanism of energy generation in the sun by observing the flux of neutrinos produced by the sun. Raymond Davis took up the challenge. He designed an experiment with a tank containing 600 tons of \( \text{CCl}_4 \) underground in the Homestake mine in South Dakota. Solar neutrinos would convert \( \text{Cl}^{37} \) to \( \text{Ar}^{37} \) at the rate of atoms/month.
The radioactive \( \text{Ar} \) atoms could then be extracted and counted. The
rate of Ar production was observed to be consistenly low compared to
the solar model prediction

\end{document}
