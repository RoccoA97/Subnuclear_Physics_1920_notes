\providecommand{\main}{../../main}
\providecommand{\figpath}[1]{\main/../lessons/#1}
\documentclass[../../main/main.tex]{subfiles}

\newdate{date}{25}{03}{2019}


\begin{document}

\chapter{Cross section of \( e^+e^- \rightarrow \mu^+\mu^- \) and \( e^+e^- \rightarrow hh \)}

\marginpar{ \textbf{Lecture 6.} \\  \displaydate{date}. \\ Compiled:  \today.}

The first is a quantum electromagnetic process and it is relatively simple to compute its cross section at first order. It is also our benchmark when we start to study the second process.





\section{\( e^+e^- \rightarrow \mu^+\mu^- \)}
The idea is now to study the cross section of the first process \( e^+e^- \longrightarrow \mu^+\mu^- \). The matrix elements for this process can be constructed by breaking the process down into components. First, the \( e^+e^- \) state is annihilated by an electromagnetic current. This current couples to a quantum state of electromagnetic excitation. Finally, this state couples to another current matrix element describing the creation of the muon pair. These passages can be drawn in a Feynman diagram in a very simple way, as in Figure \ref{L06_FDEPA}.

%TODO feynamn diagram

The intermediate photon state can be described as a Breit-Wigner resonance at zero mass. Taking the limit of zero resonance mass in the Breit-Wigner formuala, it would then contribute to the scattering amplitude by a factor:
\begin{equation}
	\frac{1}{q^2 - m^2_\mathrm{R} + \frac{i}{2} m_\mathrm{R}\Gamma_\mathrm{R}}
	\sim
	\frac{1}{q^2}
	\label{eq:L06_PAF}
\end{equation}
where \( q \) is the momentum carried by the photon from the initial to the final state. Moreover, we consider the reaction at energies large compared to the muon mass and, certainly, very far from the mass shell condition \( q^2 = 0 \) for a photon, therefore we approximate: \( m_e = m_\mu = 0\). A resonance contributing to an elementary particle reaction very far from its mass shell is called a \textbf{virtual particle}. In this case, we say that the reaction is mediated by a \textbf{virtual photon}.

So, the matrix element reads:
\begin{equation}
	\mathcal{M}(e^+e^- \longrightarrow \mu^+\mu^-)
	=
	(-e) \bra{\mu^+\mu^-} j^{\mu} \ket{0}
	\frac{1}{q^2}
	(-e) \bra{0} j_{\mu} \ket{e^+e^-}
	\label{eq:L06_EPMAME}
\end{equation}
The operator structure \( j^\mu j_\mu \) that appears in Eq. \ref{eq:L06_EPMAME} is called \textbf{current-current interaction}.



\subsection{Properties of massless spin-\( \frac{1}{2} \) fermions}
We will focus now on the properties of the massless spin-\( \frac{1}{2} \) fermions in order to evaluate Eq. \ref{eq:L06_EPMAME}.

The dynamics of fermions and the calculation of matrix elements is quite simplified in the ultrarelativistic limit, which is our case since we are considering energies so large that both the electrons and muons are moving relativistically and their masses can be neglected. In this approximation, the Dirac equation takes the form:
\begin{equation}
	i \gamma^{\mu} \partial_{\mu} \psi
	=
	0
	\label{eq:}
\end{equation}
where:
\begin{equation}
	\gamma^0
	=
	\begin{pmatrix}
		0   & \id \\
		\id & 0
	\end{pmatrix}
	\qquad
	\gamma^i
	=
	\begin{pmatrix}
		0   & \sigma_i \\
		-\sigma_i & 0
	\end{pmatrix}
	\label{eq:}
\end{equation}
It is convenient to write this representation by defining \( \sigma^\mu = (\id, \va{\sigma})^\mu \) and \( \bar{\sigma}^\mu = (\id, - \va{\sigma})^\mu \), so:
\begin{equation}
	\gamma^\mu
	=
	\begin{pmatrix}
		0   & \sigma^\mu \\
		\bar{\sigma}^\mu   & 0
	\end{pmatrix}
	\label{eq:}
\end{equation}
Moreover, we will write \( \Psi = (\psi_L, \psi_R) \), so the Dirac equation splits into:
\begin{subequations}
	\begin{align}
		i\bar{\sigma} \cdot \partial{\psi_L} &= 0	\\
		i\sigma \cdot \partial{\psi_R} &= 0
	\end{align}
	\label{}
\end{subequations}

%TODO end the part on the solutions
At the end of all the calculations, we get the solutions, with the following characteristics\footnote{We can consider the helicity \( h = \va{p} \cdot \va{s} \) to describe the solutions.}:
\begin{itemize}
	\item \( E = p > 0 \), \( s_3 = \frac{1}{2} \).
	\item \( E =-p < 0 \), \( s_3 = \frac{1}{2} \).
\end{itemize}
So we find an electron which is left-handed and a positron which is right-handed, and viceversa, a couple of right-handed electron and left-handed positron.



\subsection{Matrix element evaluation}
The first step is to evaluate the matrix element for \( e^-_R e^+_L \) and \( e^-_L e^+_R \) annihilations. In all, the process \( e^-e^+ \longrightarrow \mu^-\mu^+ \) has four amplitudes for the various spin states that are permitted by helicity conservation. All of the differential cross sections have the same structure. So, by considering that:
\begin{subequations}
	\begin{align}
		\abs{\mathcal{M}(e^-_R e^+_L \rightarrow \mu^-_R \mu^+_L)}^2
		&=
		\abs{\mathcal{M}(e^-_L e^+_R \rightarrow \mu^-_L \mu^+_R)}^2
		=
		e^4 (1 + \cos \theta)^2
		\\
		\abs{\mathcal{M}(e^-_R e^+_L \rightarrow \mu^-_L \mu^+_R)}^2
		&=
		\abs{\mathcal{M}(e^-_L e^+_R \rightarrow \mu^-_R \mu^+_L)}^2
		=
		e^4 (1 - \cos \theta)^2
	\end{align}
	\label{eq:}
\end{subequations}
we have for example, for \( e^-_R e^+_L \longrightarrow \mu^-_R \mu^+_L \):
\begin{align}
	\sigma 	&= \frac{1}{2E \cdot 2E \cdot E} \int \d{\Pi_2} \abs{\mathcal{M}}^2 \nonumber \\
	 		&= \frac{1}{2E^2_\mathrm{CM}} \frac{1}{8\pi} \int \frac{\d{\cos \theta}}{2} e^4 (1 + \cos \theta)^2
	\label{eq:}
\end{align}
and for \( e^-_R e^+_L \longrightarrow \mu^-_L \mu^+_R \):
\begin{align}
	\sigma 	&= \frac{1}{2E \cdot 2E \cdot E} \int \d{\Pi_2} \abs{\mathcal{M}}^2 \nonumber \\
	 		&= \frac{1}{2E^2_\mathrm{CM}} \frac{1}{8\pi} \int \frac{\d{\cos \theta}}{2} e^4 (1 - \cos \theta)^2
	\label{eq:}
\end{align}

With some algebra, we get the differential cross sections:
\begin{subequations}
	\begin{align}
		\dv{\sigma}{\cos \theta}	&=	\frac{\pi \alpha^2}{2E^2_\mathrm{CM}} (1 + \cos \theta)^2
		\qquad \text{for} \ e^-_R e^+_L \rightarrow \mu^-_R \mu^+_L \ \text{and} \ e^-_L e^+_R \rightarrow \mu^-_L \mu^+_R
		\\
		\dv{\sigma}{\cos \theta}	&=	\frac{\pi \alpha^2}{2E^2_\mathrm{CM}} (1 - \cos \theta)^2
		\qquad \text{for} \ e^-_R e^+_L \rightarrow \mu^-_L \mu^+_R \ \text{and} \ e^-_L e^+_R \rightarrow \mu^-_R \mu^+_L
	\end{align}
	\label{eq:}
\end{subequations}

At the end, we get the final result:
\begin{equation}
	\sigma
	=
	\frac{4\pi \alpha^2}{3E^2_\mathrm{CM}}
	\label{eq:}
\end{equation}
What is important to remember is that the cross section goes as the inverse squared of the energy in the center of mass. This is a common behaviour for electromagnetic interactions. However, at very high energies this behaviour is broken and there are corrections to consider.

How can we measure muons in a given polarization state? Actually, this is very difficult and it is not possible with the odiern technology, so we can measure an average


\end{document}
