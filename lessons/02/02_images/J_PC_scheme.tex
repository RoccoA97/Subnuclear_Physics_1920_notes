% Energy level diagrams - illustrating Hund's rule
% Author: Henri Menke
\documentclass[tikz, border=10pt]{standalone}
\usepackage{siunitx}
\usetikzlibrary{shapes.callouts}
\tikzset{
  level/.style   = { ultra thick },
  connect/.style = { dashed, red },
  notice/.style  = { draw, rectangle callout, callout relative pointer={#1} },
  label/.style   = { text width=2cm },
  trans/.style   = { dashed, blue }
}
\begin{document}
\begin{tikzpicture}
    \draw[level] node[left] at (0,0.5) {1S} (0,0) -- (2,0) node[right] at (2,0) {\( 0^{-+} \)};
    \draw[level]                            (0,1) -- (2,1) node[right] at (2,1) {\( 1^{--} \)};

    \draw[level] node[left] at (0,6.5) {2S} (0,6) -- (2,6) node[right] at (2,6) {\( 0^{-+} \)};
    \draw[level]                            (0,7) -- (2,7) node[right] at (2,7) {\( 1^{--} \)};

    \draw[level] node[left] at (4,4)   {2P} (4,4) -- (6,4) node[right] at (6,4) {\( 1^{+-} \)};

    \draw[level]                            (8,3) -- (10,3) node[right] at (10,3) {\( 0^{++} \)};
    \draw[level]                            (8,4) -- (10,4) node[right] at (10,4) {\( 1^{++} \)};
    \draw[level]                            (8,5) -- (10,5) node[right] at (10,5) {\( 2^{++} \)};

    \draw node at (5,2) {\( S = 0 \)};
    \draw node at (9,2) {\( S = 1 \)};


  %% Draw labels
  %\node[label] at (4,5.5)  {Spin-spin interaction};
  %\node[label] at (7,5.5)  {Orbit-orbit interaction};
  %\node[label] at (10,5.5) {Spin-orbit interaction};
%
  %% Draw annotations
  %\node[notice={(0.5,0.5)}, text width=1.5cm] at (2,-3) {Hunds rule \# 1};
  %\node[notice={(0,1)}] at (4,-4) {Why is triplet lower};
  %\node[notice={(0.7,0.7)}, text width=3cm] at (6,-5)
%    {Why is higher angular momentum state lower energy?};
  %\node[notice={(-0.9,0.9)}, text width=1.5cm] at (9,-5) {Hunds rule \# 2};
  %\node[notice={(-0.2,1.6)}, text width=3cm] at (11,-6.5)
%    {Why is low total angular momentum state lower in energy?};
  %\node[notice={(-0.5,0.5)}, text width=1.5cm] at (12,-5) {Hunds rule \# 3};
\end{tikzpicture}
\end{document}
