\documentclass[border=3pt]{standalone}
\usepackage{siunitx}
\usepackage[compat=1.1.0]{tikz-feynman}
\usepackage{amsmath, mathtools}
\begin{document}


\begin{minipage}[h]{0.25\linewidth}
	\centering
	\feynmandiagram [vertical=b to c] {
		i1 -- [anti fermion] a -- [anti fermion] i2,
		a -- [gluon] b
		-- [gluon, half left, edge label=\(g\), looseness=1.71] c
		-- [gluon, half left, edge label=\(g\), looseness=1.71] b,
		c -- [gluon] d,
		f1 -- [fermion] d -- [fermion] f2,
	};
\end{minipage}
\( \qquad = \qquad \)
\begin{minipage}[h]{0.25\linewidth}
	\centering
	\feynmandiagram [vertical=b to c] {
		i1 -- [anti fermion] a -- [anti fermion] i2,
		a -- [scalar] b
		-- [gluon, half left, edge label=\(g\), looseness=1.71] c
		-- [gluon, half left, edge label=\(g\), looseness=1.71] b,
		c -- [scalar] d,
		f1 -- [fermion] d -- [fermion] f2,
	};
\end{minipage}
\( \qquad + \qquad \)
\begin{minipage}[h]{0.25\linewidth}
	\centering
	\feynmandiagram [vertical=b to c] {
		i1 -- [anti fermion] a -- [anti fermion] i2,
		a -- [scalar] b
		-- [gluon, half left, edge label=\(g\), looseness=1.71] c
		-- [scalar] b,
		c -- [scalar] d,
		f1 -- [fermion] d -- [fermion] f2,
	};
\end{minipage}

\end{document}
