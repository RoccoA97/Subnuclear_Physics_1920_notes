\providecommand{\main}{../../main}
\providecommand{\figpath}[1]{\main/../lessons/#1}
\documentclass[../../main/main.tex]{subfiles}

\newdate{date}{28}{04}{2020}


\begin{document}

\marginpar{ \textbf{Lecture 14.} \\  \displaydate{date}. \\ Compiled:  \today.}

When the Higghs was proposed, Weinberg and Salam showed that the required mass of the vector bosons and of the fermions can be acquired by the interaction with the \( H \) field. The potential of this field is:
\begin{equation}
	V(\varphi)
	=
	-\mu^2 \abs{\varphi}^2 + \lambda \qty(\abs{\varphi}^2)^2
	\label{eq:}
\end{equation}
The minimum of the potential is obtained by setting to zero the first derivative with respect to \( \varphi \) of the potential:
\begin{equation}
	0
	=
	-2\mu^2 \varphi + 4 \lambda \varphi \abs{\varphi}^2
	\Longrightarrow
	\abs{\varphi}^2
	=
	\frac{\mu^2}{2\lambda}
	\label{eq:}
\end{equation}
If we compute the \( H \) field vacuum expectation value, we find:
\begin{equation}
	v
	=
	\sqrt{2} \left\langle \abs{\varphi} \right\rangle
	=
	\frac{\mu}{\sqrt{\lambda}}
	\label{eq:}
\end{equation}
It is different from zero and it spontaneously breaks the \( SU(2) \times U(1) \) gauge symmetry.

In the theory we need to introduce two coupling constants, \( g \) and \( g' \), related to the two gauge groups. Putting all together the previous facts, it is possible to show that:
\begin{itemize}
	\item One of the 4 bosons of theory remains massless and it is the photon.
	\item Two bosons are charged with mass, namely:
		\begin{equation}
			M_W = g \frac{v}{\sqrt{2}}
			\label{eq:}
		\end{equation}
	\item One boson is neutral and massive, with mass:
		\begin{equation}
			M_{Z^0} = \sqrt{g'^2 + g^2} \frac{v}{2}
			\label{eq:}
		\end{equation}
	\item If we define the Weinberg angle \( \theta_W \), we have:
		\begin{equation}
			\tan \theta_W = \frac{g'}{g}
			\label{eq:}
		\end{equation}
	\item It holds:
		\begin{align}
			M_W &= M_Z \cos \theta_W	\\
			e &= g \sin \theta_W = g' \cos \theta_W
		\end{align}
\end{itemize}

\( W^{\pm} \) and \( Z^0 \) bosons couple to quarks and leptons. In order to describe these couplings two quantum numbers are needed: the \textbf{weak isospin \( I \)} and the \textbf{hypercharge \( Y \)}. The electric charge is then related to these quantities by:
\begin{equation}
	Q
	=
	I_3 + \frac{Y}{2}
	\label{eq:}
\end{equation}
Experimentally, we see that left-handed particles couple to \( W^{\pm} \) bosons, but right-handed particles have not this behaviour. Quarks and leptons are grouped:
\begin{itemize}
	\item Left-handed \( \Longrightarrow \) doublet \( \Longrightarrow I = \frac{1}{2} \)
	\item Right-handed \( \Longrightarrow \) singlet \( \Longrightarrow I = 0 \)
\end{itemize}
Therefore the correct representation is:
\begin{equation}
	\begin{pmatrix}
		\nu_L \\
		e^-_L
	\end{pmatrix}, e^-_R
	\qquad
	\begin{pmatrix}
		u_L \\
		d_L
	\end{pmatrix}, u_R, d_R
	\label{eq:}
\end{equation}

\end{document}
