\providecommand{\main}{../../main}
\providecommand{\figpath}[1]{\main/../lessons/#1}
\documentclass[../../main/main.tex]{subfiles}

\newdate{date}{26}{05}{2020}


\begin{document}

\chapter{New Physics}

\marginpar{ \textbf{Lecture 22.} \\  \displaydate{date}. \\ Compiled: \today.}

\section{Neutrino and Standard Model}
The neutrino was introduced by Pauli in the study of \( \beta \) decay. This requires that the mass of the electron neutrino, at
least, is very small. A bound on the mass of \( \nu_e \) can be set by studying in particular the endpoint of the electron energy distribution. If we consider the rate for the \( \beta \) decay of a nucleus \( A \) to \( B \), it has the following form:
\begin{equation}
	\Gamma(A \rightarrow Be^-\bar{\nu})
	=
	\frac{1}{2m_A}
	\int \frac{\d{^3p_B} \d{^3p_e} \d{^3 p_{\nu}}}{(2\pi)^9 2E_B 2E_e 2E_{\nu}}
	(2\pi)^4 \delta^{(4)}(p_A - p_B - p_e - p_{\nu})
	\abs{\mathcal{M}}^2
	\label{eq:}
\end{equation}
Since we are interested to the endpont of \( e^- \) spectrum, we will have that \( \abs{\mathcal{M}}^2 \) is approximately constant in our study. \( A \) and \( B \) are very hevy compared to their mass difference (tipically of few \( \si{MeV} \)). So, it is also a good approximation to assume that the final nucleus \( B \) takes up the recoil momentum, so that the directions of the electron and neutrino are uncorrelated. In this limit, the energies of the final electron and neutrino sum to:
\begin{equation}
	E_e + E_{\nu}
	=
	m(A) - m(B)
	=
	\Delta m_{AB}
	\label{eq:}
\end{equation}
Then, the rate becomes:
\begin{equation}
	\Gamma(A \rightarrow Be^-\bar{\nu})
	=
	\frac{1}{2m_A} \frac{1}{(2\pi)^5 2m_B}
	\int \frac{\d{p_e} p_e^2}{2E_e}
	\int \frac{\d{p_{\nu}} p_{\nu}^2}{2E_{\nu}}
	\delta(\Delta m_{AB} - E_e - E_{\nu})
	\abs{\mathcal{M}}^2
	\label{eq:}
\end{equation}
Using:
\begin{align*}
	\d{p_e}p_e &= \d{E_e} E_e	\\
	\d{p_{\nu}}p_{\nu} &= \d{E_{\nu}} E_{\nu}
\end{align*}
we can write the decay rate as:
\begin{equation}
	\Gamma
	\sim
	\int_{m_e}^{\Delta m_{AB}} \d{E_e}
	\int_{0}^{\Delta m_{AB} - E_e} \d{E_{\nu}} p_{\nu}
	\label{eq:}
\end{equation}
Assuming that the neutrino has zero mass, this gives:
\begin{equation}
	\dv{\Gamma}{E_e}
	\sim
	(\Delta m_{AB} - E_e)^2
	\label{eq:}
\end{equation}
This energy distribution is conventionally represented by a \textbf{Kurie plot}, plotting the square root of the event rate as a function of the electron energy. This should be a straight line for a zero mass neutrino. However, if the neutrino is massive, the plott falls off the kinematic endpoint:
\begin{equation}
	E_e
	=
	\Delta m_{AB} - m_{\nu}
	\label{eq:}
\end{equation}

%TODO (20.7) p.312
Measurements of \( \beta \) decay exclude \( \nu_e \) masses of more than a few \( \si{eV} \). However, \( \beta \) decay is not the most adequate way to measure \( m_{\nu} \) mass. The \( \beta \) electron can lose an energy of order \( \si{eV} \) when it exits the atom, and it loses \( \si{eV/mm} \) in traversing material. These energy losses must be accounted for in the interpretation of the electron energy distribution. The most accurate measure up to 2019 was done by Mainz and Troitsk and it gives the limit:
\begin{equation}
	m_{\nu_e}
	<
	2.05 \ \si{eV}
	\label{eq:}
\end{equation}
In September 2019, the KATRIN experiment analyzed the decay of gaseous Tritium. From the fit of the collected data, it was found a new limit for the mass of electronic neutrino at 90\% CL:
\begin{equation}
	m_{\nu_e}
	<
	1.1 \ \si{eV}
	\label{eq:}
\end{equation}
%sensitivity of 0.2 eV on neutrino mass from KATRIN
%TODO add plots

Limits on \( \nu_{\mu} \) and \( \nu_{\tau} \) mass are weaker with respect to the \( \nu_e \) limit. For them, we can study:
\begin{align}
	\pi^+ &\longrightarrow \mu^+ \nu_{\mu}	\\
	\tau^- &\longrightarrow 2\pi^-\pi^+ \nu_{\tau}
	\label{eq:}
\end{align}
and we find respectively the following limits on the masses:
\begin{align}
	m(\nu_{\mu}) &< 0.19 \ \si{MeV}	\\
	m(\nu_{\tau})&< 18.2 \ \si{MeV}
\end{align}

With a cosmology argument, if we have a massive \( \nu \) that moves relativistically in the early universe, \( \nu \) would transfer energy and smear out cosmic structure, giving an observably different distribution of clusters of galaxies if the neutrino masses are sufficiently large. The absence of this effect gives a bound estimated to be:
\begin{equation}
	\sum_{i=1,2,3} m_{\nu_i}
	<
	0.23 \ \si{eV}
	\label{eq:}
\end{equation}





\bigskip
Neutrino masses are thus very small compared to the weak interaction mass scale, sufficiently small that it is unclear how they can be observed. To understand the evidence for neutrino mass, we need to develop further the theory of neutrino massses within the Standard Model.

Let's assume the gauge invariance and a \( SU(2) \times U(1) \) symmetry. Neutrino masses can arise in one of the following two ways:
\begin{itemize}
	\item The simplest mechanism is to assume that there exist right-handed neutrinos that couple to the left-handed neutrinos through Yukawa couplings. This translates in adding to the Standard Model Lagrangian a term:
	\begin{equation}
		\Delta \mathcal{L}
		=
		- y^{ij}_{\nu} L^{\dag i}_a \varepsilon_{ab} \varphi^*_b \nu^j_R
		+ \text{h.c.}
		\label{eq:}
	\end{equation}
	similar to the \( u \) quark mass term that couples to \( W^{\pm} \). This term can be treated in the same way the quark and lepton mass terms are treated. However, this is not appropriate. In elementary particle reactions, neutrinos are typically emitted at MeV or higher energies, at which effects of eV-scale masses are unimportant. Therefore it is most convenient to retain our earlier convention that the left-handed neutrinos are described in the basis that diagonalizes their weak interactions. We then trat the new term by making the change of variables:
	\begin{equation}
		L^i \longrightarrow U^{(e)}_{L_{ij}} L^j
		\label{eq:}
	\end{equation}
	This transforms:
	\begin{equation}
		y_{\nu} \longrightarrow y'_{\nu} = U^{(e)\dag}_L y_{\nu}
		\label{eq:}
	\end{equation}
	This transformation diagonalizes the charged lepton Yukawa matrix but does not necessarily diagonalize the neutrino Yukawa matrix. This basis for neutrino states will be denoted as the basis of flavor eigenstates.
	%TODO

	\( \dots  \)

	The mixing matrix \( U^{(\nu)}_R \) is called the Pontecorvo-Maki-Nakagawa-Sakata or PMNS matrix and is more commonly notated \( V \) or \( V_{PMNS} \).

	\item There is another way to add neutrino masses to the Standard Model that is consistent with Lorentz invariance and \( SU(2) \times U(1) \).
	%TODO

	\( \dots  \)

	When the Higgs field \( \varphi \) acquires an expectation value and breaks \( SU(2) \times U(1) \), (20.21) leads to a mixing of the \( \nu_L \) states with their antiparticles \( \bar{\nu_R} \), generating masses given by the eigenvalues of:
	\begin{equation}
		m_{ij}
		=
		\mu_{ij} \frac{v^2}{2}
		\label{eq:}
	\end{equation}
	This mass term, resulting from particle-antiparticle mixing, is called a \textbf{Majorana mass term}.
	%TODO

	\( \dots  \)

	This is called the \textbf{seesaw mechanism} for generating small neutrino masses. It produces small masses by a modification of the theory at very high energies.
	%TODO

	\( \dots  \)

	The Majorana mass term gives a new weak interaction process: the double \( \beta \) decay. These are some og the rarest physical processes known. For example, the EXO experiment measure:
	\begin{equation}
		\tau(\text{Xe}^{136})
		=
		2 \cdot 10^{21} \ \si{yr}
		\label{eq:}
	\end{equation}
	If the neutrino \( \bar{\nu}_e \) has a lepton-number violating Majorana mass term, then also the decay proces:
	\begin{equation}
		\text{Xe}^{136}
		\longrightarrow
		\text{Ba}^{136} + e^-e^-
		\label{eq:}
	\end{equation}
	is allowed, with no final-state neutrinos. The rate of this decay is expected to be small even in comparison to the lifetime of \( \text{Xe}^{136} \) previously given.
	Check EXO and CUORE experiments. This and similar decays are being intensively searched for, but none has yet been observed.
\end{itemize}





\section{The \( V_{PNMS} \) effect}
In order to explain the effects of \( V_{PNMS} \) matrix, we assume to have just two neutrinos. Them:
\begin{equation}
	V
	=
	\begin{pmatrix}
		\cos\theta & -\sin\theta \\
		\sin\theta &  \cos\theta
	\end{pmatrix}
	\label{eq:}
\end{equation}
So:
\begin{equation}
	\begin{pmatrix}
		\nu_{\mu}	\\
		\nu_{e}
	\end{pmatrix}
	=
	V
	\begin{pmatrix}
		\nu_1	\\
		\nu_2
	\end{pmatrix}
	=
	\begin{pmatrix}
		\nu_1 \cos\theta - \nu_2 \sin\theta \\
		\nu_1 \sin\theta + \nu_2 \cos\theta
	\end{pmatrix}
	\label{eq:}
\end{equation}
So we get the time evolution as for the \( B \) meson mixing:
\begin{equation}
	\nu_{1,2}(t)
	=
	\nu_{1,2}(0) e^{-i E_{1,2} t}
	\label{eq:}
\end{equation}
Considering the process \( \pi^+ \rightarrow \mu^+ \nu_{\mu} \), we have:
\begin{equation}
	P(\nu_{\mu} \rightarrow \nu_{\mu}(t))
	=
	\abs{\cos^2\theta e^{-i \dots} + \dots}^2 %TODO
	\label{eq:}
\end{equation}
At the end of these calculations, we find:
\begin{equation}
	P(\nu_{\mu} \rightarrow \nu_{\mu})
	=
	1 - \sin^2(2\theta) \sin^2 \\qty[(m_1 - m_2)^2 \frac{L}{4E}]
	\label{eq:}
\end{equation}
where \( L \) is the distance travelled by \( \nu_{\mu} \). Moreover:
\begin{equation}
	P(\nu_{\mu} \rightarrow \nu_{e})
	=
	1 - P(\nu_{\mu} \rightarrow \nu_{\mu})
	\label{eq:}
\end{equation}
\begin{equation}
	P(\nu_{\mu} \rightarrow \nu_{\mu})
	=
	1 - \sin^2(2\theta) \sin^2 \\qty[1.2 + \frac{\Delta m^2 L}{E}]
	\label{eq:}
\end{equation}
\( L \) is in meters, \( \Delta m^2 \) is in \( \si{eV^2} \) and \( E \) is in \( \si{MeV} \).


\end{document}
