\documentclass{article}
\usepackage{amsmath} % for \text
\usepackage{tikz}
\tikzset{>=latex} % for LaTeX arrow head
%\definecolor{mylightred}{RGB}{255,200,200}
%\definecolor{mylightblue}{RGB}{172,188,63}
%\definecolor{mylightgreen}{RGB}{150,220,150}

% split figures into pages
\usepackage[active,tightpage]{preview}
\PreviewEnvironment{tikzpicture}
\setlength\PreviewBorder{1pt}%

% vertical custom shading
%https://tex.stackexchange.com/questions/191735/is-it-possible-to-define-the-position-of-top-bottom-and-middle-color-in-fill
\makeatletter
\tikzset{vertical custom shading/.code={%
  \pgfmathsetmacro\tikz@vcs@middle{#1}
  \pgfmathsetmacro\tikz@vcs@bottom{\tikz@vcs@middle/2}
  \pgfmathsetmacro\tikz@vcs@top{(100-\tikz@vcs@middle)/2+\tikz@vcs@middle}
  \pgfdeclareverticalshading[tikz@axis@top,tikz@axis@middle,tikz@axis@bottom]{newaxis}{100bp}{%
    color(0bp)=(tikz@axis@bottom);
    color(\tikz@vcs@bottom bp)=(tikz@axis@bottom);
    color(\tikz@vcs@middle bp)=(tikz@axis@middle);
    color(\tikz@vcs@top bp)=(tikz@axis@top);
    color(100bp)=(tikz@axis@top)}
    \pgfkeysalso{/tikz/shading=newaxis}
  }
}
\makeatother

\begin{document}



% arrow style
\usetikzlibrary{decorations.markings}
\tikzset{->-/.style={decoration={markings,
                               mark=at position .7 with {\arrow{>}}},
                               postaction={decorate}}}

% photon
\usetikzlibrary{decorations.pathmorphing}
\tikzset{photon/.style={decorate, decoration={snake,segment length=5,amplitude=1.1}}}

% cone style
\usetikzlibrary{shadows.blur}
\tikzset{mycone/.pic={
       \shadedraw[top color=white, bottom color=black!50,shading angle=50,vertical custom shading=10]
                    (-0.4,0.99) -- (0,0) -- (0.4,0.99);
       \shadedraw[top color=white, bottom color=black!20,shading angle=90]
                    (0,1) ellipse (0.4 and 0.12);}};



% TAU DECAY MODES
\begin{tikzpicture}[scale=0.9]

  \large
%  % arrow style
%  \usetikzlibrary{decorations.markings}
%  \tikzset{->-/.style={decoration={markings,
%                                   mark=at position .7 with {\arrow{>}}},
%                                   postaction={decorate}}}

  % photon
  \usetikzlibrary{decorations.pathmorphing}
  \tikzset{photon/.style={decorate, decoration={snake,segment length=5,amplitude=1.1}}}

  % cone style
  \usetikzlibrary{shadows.blur}
  \tikzset{mycone/.pic={
           \shadedraw[top color=white, bottom color=black!50,shading angle=50,vertical custom shading=10]
                        (-0.4,0.99) -- (0,0) -- (0.4,0.99);
           \shadedraw[top color=white, bottom color=black!20,shading angle=90]
                        (0,1) ellipse (0.4 and 0.12);}};

  \def\R{10}             % inner radius
  \def\hca{16}           % half central angle
  \def\hcl{\R*sin(\hca)} % half chord length c = 2Rsin(theta/2)
  \def\tracker{3}        % tracker
  \def\ECAL{4.8}         % ECAL
  \def\HCAL{7.2}         % HCAL
  \def\hclmax{(\R+\HCAL)*sin(\hca)} % mac

  % detectors
  \foreach \r in {\tracker,\ECAL,\HCAL}{
    \draw[thick] (0,\r) arc (90:90-\hca:\R+\r)
                 (0,\r) arc (90:90+\hca:\R+\r);
  }
  \node[left=0,above left] at ({1.1*\hclmax},{0.10*\tracker}) {tracker};
  \node[left=0,above left] at ({1.1*\hclmax},\tracker) {ECAL};
  \node[left=0,above left] at ({1.1*\hclmax},\ECAL) {HCAL};

  % DM 1
  \begin{scope}[shift={(0,0)}, rotate=0]
    \node[below] at (0,0) {$\tau^\pm$};
    \node[right=8pt,below=18] at (0,0) {$\tau^\pm\to\rho^\pm\nu_\tau\to\pi^\pm\pi^0\nu_\tau$};
    \draw[->-,thick]                   % tau
      (0,0) -- (0,0.7) coordinate (tau);
    \draw[->-,dotted,thick,rotate=-45] % neutrinos
      (tau) -- +(0,1.5) node[above right] {$\nu_\tau$};
    \draw[->-,       thick,rotate= 15] % rho
      (tau) -- +(0,0.7) coordinate (rho) node[midway,left] {$\rho^\pm$};
    \draw[->-,       thick,rotate= 24] % pion
      (rho) -- +(0,3.4) node[left=15,above=38] {$\pi^\pm$}
      pic[rotate=22,scale=1.2] {mycone};
    \draw[->-,densely dashed,thick,rotate=-18] % pion0
      (rho) -- +(0,0.7) coordinate (pion0) node[midway,right] {$\pi^0$};
    \draw[photon, rotate= 10] % photon 1
      (pion0) -- +(0,0.9) node[left=8,above=32] {$\gamma$}
      pic[rotate=11.5,xscale=0.8] {mycone};
    \draw[photon,rotate=-30] % photon 2
      (pion0) -- +(0,1.0) node[right=15,above=30] {$\gamma$}
      pic[rotate=-24,xscale=0.8] {mycone};
  \end{scope}

\end{tikzpicture}




\end{document}
